\documentclass[defesa,oneside,brazilian,english]{ppginf}

\include{packages}

\makeatletter
\g@addto@macro\@floatboxreset{\centering}
\makeatother
% Language setting
% Replace `english' with e.g. `spanish' to change the document language
%\usepackage[brazilian]{babel}

% Set page size and margins
% Replace `letterpaper' with `a4paper' for UK/EU standard size
%\usepackage[letterpaper,top=2cm,bottom=2cm,left=3cm,right=3cm,marginparwidth=1.75cm]{geometry}

% Useful packages
\usepackage{amsmath}
\usepackage{graphicx}
\usepackage{bm}
\usepackage{tikz}
\usetikzlibrary{arrows.meta}
%\usepackage{float}


%\usepackage[colorlinks=true, allcolors=blue]{hyperref}

\begin{document}



\title{Analise crítica dos limites do Blockchain em relacão a escalabilidade e capacidade de processamento de transacões em grande escala}
\author{Kokouvi Hola Kanyi Kodjovi}
\advisor{Eduardo Todt}

% instituição
\IfLanguageName{brazilian}
  { \instit{UFPR}{Universidade Federal do Paraná} }
% a Bib/UFPR exige que tudo seja em português, exceto o título :-(
%  { \instit{UFPR}{Federal University of Paraná} }
  { \instit{UFPR}{Universidade Federal do Paraná} }

% área de concentração (default do PPGInf, não mudar)
\IfLanguageName{brazilian}
  { \field{Ciência da Computação} }
% a Bib/UFPR exige que tudo seja em português, exceto o título :-(
%  { \field{Computer Science} }
  { \field{Ciência da Computação} }

% data (só o ano)
% \date{2023}

% local
\IfLanguageName{brazilian}
  { \local{Curitiba PR} }
% a Bib/UFPR exige que tudo seja em português, exceto o título :-(
%  { \local{Curitiba PR - Brazil} }
  { \local{Curitiba PR} }

% imagem de fundo da capa (se não desejar, basta comentar)
% \coverimage{0-iniciais/fundo-capa.jpg}

%=====================================================

%% Descrição do documento (obviamente, descomentar somente UMA!)

% Por exigência da biblioteca da UFPR, a descrição do documento deve ser
% em português, mesmo em documentos em outras línguas. Vá entender...

% tese de doutorado
%\descr{Tese apresentada como requisito parcial à obtenção do grau de Doutor em Ciência da Computação no Programa de Pós-Graduação em Informática, Setor de Ciências Exatas, da Universidade Federal do Paraná}

% exame de qualificação de doutorado
%\descr{Documento apresentado como requisito parcial ao exame de qualificação de Doutorado no Programa de Pós-Graduação em Informática, Setor de Ciências Exatas, da Universidade Federal do Paraná}

% dissertação de mestrado
%\descr{Dissertação apresentada como requisito parcial à obtenção do grau de Mestre em Informática no Programa de Pós-Graduação em Informática, Setor de Ciências Exatas, da Universidade Federal do Paraná}

% exame de qualificação de mestrado
%\descr{Documento apresentado como requisito parcial ao exame de qualificação de Mestrado no Programa de Pós-Graduação em Informática, Setor de Ciências Exatas, da Universidade Federal do Paraná}

% trabalho de conclusão de curso
\descr{Trabalho do conclusão do curso de bacharelado  em  Ciências de computação }
% trabalho de disciplina
%\descr{Trabalho apresentado como requisito parcial à conclusão da disciplina XYZ no Curso de Bacharelado em XYZ, Setor de Ciências Exatas, da Universidade Federal do Paraná}

% doctorate thesis
%\descr{Thesis presented as a partial requirement for the degree of Doctor in Computer Science in the Graduate Program in Informatics, Exact Sciences Sector, of the Federal University of Paraná, Brazil}

% doctorate qualification
%\descr{Document presented as a partial requirement for the doctoral qualification exam in the Graduate Program in Informatics, Exact Sciences Sector, of the Federal University of Paraná, Brazil}

% MSc dissertation
%\descr{Dissertation presented as a partial requirement for the degree of Master of Sciences in Informatics in the Graduate Program in Informatics, Exact Sciences Sector, of the Federal University of Paraná, Brazil.}

% MSc qualification
%\descr{Document presented as a partial requirement for the Master of Sciences qualification exam in the Graduate Program in Informatics, Exact Sciences Sector, of the Federal University of Paraná, Brazil}

%=====================================================

% define estilo das páginas iniciais (capas, resumo, sumário, etc)

\frontmatter
\pagestyle{frontmatter}

% produz capa e folha de rosto
\titlepage
%\maketitle
\begin{resumo}


    O Blockchain é uma tecnologia inovadora com o potencial de transformar diversos setores, desde serviços financeiros até cadeias de suprimentos. No entanto, sua adoção em larga escala tem enfrentado desafios significativos relacionados à escalabilidade e à capacidade de processamento de transações. A escalabilidade refere-se à capacidade do Blockchain de lidar com um grande número de transações e usuários sem comprometer a eficiência. À medida que mais participantes se juntam à rede, os tempos de confirmação de transações aumentam, tornando o Blockchain ineficiente em grande escala. O aumento do volume de transações também coloca pressão nos recursos computacionais necessários para validá-las. Outros desafios incluem o armazenamento contínuo de dados, à medida que o Blockchain cresce, e questões de governança e coordenação à medida que a rede envolve múltiplas organizações. Pesquisadores e desenvolvedores têm explorado soluções, como Blockchains de segunda camada, novos algoritmos de consenso (como Prova de Participação e Prova de Autoridade), e técnicas de dimensionamento horizontal. Este estudo aborda minuciosamente a problemática da escalabilidade e o desempenho do Blockchain, oferecendo uma análise aprofundada sobre como a escalabilidade representa um desafio premente nas redes blockchain. Além disso, são explorados os avanços tecnológicos que visam aprimorar e solucionar essa questão, destacando a importância de superar os obstáculos inerentes à expansão eficiente dessas redes.A escalabilidade do Blockchain é fundamental, e apesar dos desafios, o potencial disruptivo do Blockchain continua a atrair investimentos e inovações. À medida que a pesquisa e a colaboração avançam, soluções promissoras são esperadas para permitir que o Blockchain atinja seu pleno potencial em diversos setores.
    
    \end{resumo}

\begin{abstract}

% The Blockchain is an innovative technology with the potential to transform various sectors, from financial services to supply chains. However, its adoption on a large scale has faced significant challenges related to scalability and transaction processing capacity. Scalability refers to the Blockchain's ability to handle a large number of transactions and users without compromising efficiency. As more participants join the network, transaction confirmation times increase, making the Blockchain inefficient on a large scale. The growing volume of transactions also puts pressure on the computational resources needed to validate them. Other challenges include continuous data storage as the Blockchain expands, and governance and coordination issues as the network involves multiple organizations. Researchers and developers have explored solutions such as second-layer Blockchains, new consensus algorithms (such as Proof of Stake and Proof of Authority), and horizontal scaling techniques.
% In summary, Blockchain scalability is crucial, and despite the challenges, the disruptive potential of Blockchain continues to attract investments and innovations. As research and collaboration progress, promising solutions are expected to enable Blockchain to realize its full potential across various sectors.
Blockchain is an innovative technology with the potential to transform various sectors, from financial services to supply chains. However, its widespread adoption has faced significant challenges related to scalability and transaction processing capacity. Scalability refers to the ability of Blockchain to handle a large number of transactions and users without compromising efficiency. As more participants join the network, transaction confirmation times increase, making Blockchain inefficient on a large scale. The growing transaction volume also puts pressure on the computational resources required to validate them. Other challenges include continuous data storage as Blockchain expands, and governance and coordination issues as the network involves multiple organizations. Researchers and developers have explored solutions, such as second-layer Blockchains, new consensus algorithms (such as Proof of Stake and Proof of Authority), and horizontal scaling techniques. This study thoroughly addresses the scalability issues and performance of Blockchain, providing a detailed analysis of how scalability poses a pressing challenge in blockchain networks. Additionally, technological advances aimed at improving and resolving this issue are explored, emphasizing the importance of overcoming inherent obstacles to the efficient expansion of these networks. Blockchain scalability is crucial, and despite the challenges, the disruptive potential of Blockchain continues to attract investments and innovations. As research and collaboration progress, promising solutions are expected to enable Blockchain to reach its full potential in various sectors.
\end{abstract}
 %=====================================================

% lista de acrônimos (siglas e abreviações)

\begin{listaacron}

\begin{longtable}[l]{p{0.2\linewidth}p{0.7\linewidth}}

PoW & Proof of Work \\
TPS & Transações Por Segundo \\
HLF & Hyperledger Fabric \\
YMSC & Yet Another Scalability Challenge \\
PBFT & Practical Byzantine Fault Tolerance \\
IOHeavy & Input/Output Heavy \\
PCC & Prova de Conjectura de Collatz (Collatz Conjecture Proof)\\

\end{longtable}
\end{listaacron}

%=====================================================

% \begin{listaacron}

% \begin{longtable}[l]{p{0.2\linewidth}p{0.7\linewidth}}
% AI & Artificial Intelligence \\
% CNN & Convolutional Neural Network \\
% DMD & Driver Monitoring Dataset \\
% DT & Decision Tree \\
% GNB & Gaussian Naive Bayes \\
% KNN & K-Nearest Neighbours \\
% LR & Looking Road \\
% ML & Machine Learning \\
% NLR & Not Looking Road \\
% RBF & Radial Basis Function \\
% RF & Random Forest \\
% RGB & Red, Green and Blue \\
% SVC & Support Vector Classifier \\
% SVM & Support Vector Machine \\
% UFPR & Universidade Federal do Paraná\\
% VRI & Vision, Robotics and Imaging \\

% \end{longtable}
% \end{listaacron}		% acrônimos, deve ser preenchida à mão
\listoffigures				% figuras
%\clearpage
\listoftables	

\tableofcontents			% sumário

%=====================================================

% define estilo do corpo do documento (capítulos e apêndices)
\mainmatter
\pagestyle{mainmatter}

% inclusao de cada capítulo, alterar a gosto (do professor de Metodologia)

\chapter{Introdução}
\label{chapter_intro}


% A tecnologia blockchain tem revolucionado a forma como realizamos transações, assinaturas, contratos, etc., promovendo a descentralização e reduzindo a dependência de aprovações institucionais para validação \cite{Blockchain}. Ela oferece um registro distribuído de transações, compartilhado e verificado por uma rede de computadores  \cite{IBM}. Essas transações são organizadas em blocos, formando uma cadeia de informações imutáveis e protegidas contra adulteração \cite{Immutability}.

% Os primórdios do blockchain remontam à década de 1990, quando Stuart e W. Scott Stornetta introduziram o conceito de "árvore de carimbos de tempo" como uma solução para garantir a integridade de registros digitais. Contudo, foi somente em 2008 que a tecnologia ganhou notoriedade com a publicação do artigo "Bitcoin: A Peer-to-Peer Electronic Cash System" por Satoshi Nakamoto, descrevendo o funcionamento do Bitcoin, a primeira criptomoeda baseada em blockchain \cite{Origem-Blockchain}.

% Apesar de suas realizações e inovações, o blockchain enfrenta desafios significativos, como escalabilidade, custo de transações, interoperabilidade, privacidade e segurança  \cite{Scalability}. A escalabilidade, em particular, é um dos obstáculos mais prementes. A maioria das redes blockchain possui limitações na quantidade de transações processadas em um determinado período, o que pode dificultar sua adoção em áreas que demandam alto volume de transações, como sistemas de pagamento em larga escala \cite{Buterin2013}.

% A escalabilidade é frequentemente avaliada em termos de Transações por Segundo (TPS), uma métrica que avalia a capacidade de uma rede blockchain de processar transações em um segundo. O tamanho dos blocos e os algoritmos de consenso, como o Proof-of-Work (PoW), desempenham um papel crítico na escalabilidade. Blocos maiores podem aumentar a capacidade, mas também podem centralizar a rede e exigir mais recursos de armazenamento. Além disso, a privacidade das transações e os protocolos complexos também afetam a capacidade de processamento de transações \cite{Proof-of-Work}.

% Esta análise crítica se concentra nos desafios da escalabilidade e da capacidade de processamento em larga escala das blockchains. Examina as limitações técnicas, como armazenamento, tempos de confirmação, processamento de nós e eficiência energética . Além disso, discute as questões de governança e adoção em grande escala.

% Para embasar essa análise crítica, apresentamos diversos pontos de vista sobre os desafios da escalabilidade no contexto das blockchains. Destacamos as limitações atuais e exploramos soluções propostas pela comunidade acadêmica e pela indústria .

% A tecnologia blockchain tem redefinido fundamentalmente a maneira como conduzimos transações, assinamos contratos e realizamos uma variedade de operações, eliminando a necessidade de aprovações centralizadas para validação \cite{Blockchain}. Essa inovação estabelece um registro distribuído de transações, verificado por uma rede de computadores e organizado em blocos, criando uma cadeia de informações imutáveis e resistentes à adulteração \cite{IBM, Immutability}.

% Embora as origens do blockchain remontem à década de 1990, quando Stuart e W. Scott Stornetta apresentaram a "árvore de carimbos de tempo" para garantir a integridade de registros digitais, foi apenas em 2008 que a tecnologia ganhou destaque com a publicação do artigo seminal de Satoshi Nakamoto sobre o Bitcoin \cite{Origem-Blockchain}.

% Apesar das conquistas e inovações, o blockchain enfrenta desafios cruciais, incluindo escalabilidade, custos de transações, interoperabilidade, privacidade e segurança \cite{Scalability}. A escalabilidade, em particular, emerge como um desafio proeminente, limitando a quantidade de transações que as redes blockchain podem processar, o que impacta sua aplicabilidade em setores de alto volume, como sistemas de pagamento em larga escala \cite{Buterin2013}.

% A avaliação da escalabilidade, geralmente medida em Transações por Segundo (TPS), destaca a importância do tamanho dos blocos e dos algoritmos de consenso, como o Proof-of-Work (PoW), na eficiência da rede. Enquanto blocos maiores podem aumentar a capacidade, há o risco de centralização e demanda por mais recursos de armazenamento. Além disso, considerações sobre privacidade e complexidade dos protocolos também influenciam a capacidade de processamento de transações \cite{Proof-of-Work}.

% Este trabalho tem como objetivo explorar os limites do Blockchain em relação à escalabilidade e capacidade de processamento de transações em grandes escalas, utilizando métodos sistemáticos de análise de literatura. O Blockchain, apesar de ser uma tecnologia revolucionária com grande potencial para transformar diversos setores, apresenta alguns desafios que podem afetar sua eficiência em larga escala.
% Um dos principais desafios do Blockchain é a escalabilidade, que se refere à capacidade da tecnologia de lidar com um grande número de transações em um curto período de tempo. Isso ocorre porque cada transação é validada por um grande número de nós na rede, o que pode resultar em um gargalo de processamento de dados. Além disso, a capacidade de armazenamento de dados também pode ser um desafio, especialmente em Blockchains públicos, onde todas as transações são armazenadas em cada nó da rede.
% Esta análise crítica concentra-se nos desafios inerentes à escalabilidade e capacidade de processamento em larga escala nas blockchains. Examina aspectos técnicos, como armazenamento, tempos de confirmação, processamento de nós e eficiência energética, enquanto explora questões de governança e adoção generalizada.


% Para embasar essa análise crítica, apresentamos uma variedade de perspectivas sobre os desafios de escalabilidade no contexto das blockchains, destacando limitações atuais e explorando soluções propostas pela comunidade acadêmica e pela indústria. Essa abordagem busca informar e orientar futuros desenvolvimentos na busca por soluções mais eficazes e sustentáveis para os desafios enfrentados pela tecnologia blockchain.

A tecnologia blockchain redefine de maneira fundamental a condução de transações, a assinatura de contratos e a realização de diversas operações, eliminando a necessidade de aprovações centralizadas para validação \cite{Blockchain}. Essa inovação estabelece um registro distribuído de transações, verificado por uma rede de computadores e organizado em blocos, criando uma cadeia de informações imutáveis e resistentes à adulteração \cite{IBM, Immutability}.

Embora as origens do blockchain remontem à década de 1990, quando Stuart e W. Scott Stornetta apresentaram a "árvore de carimbos de tempo" para garantir a integridade de registros digitais, foi apenas em 2008 que a tecnologia ganhou destaque com a publicação do artigo seminal de Satoshi Nakamoto sobre o Bitcoin \cite{Origem-Blockchain}.

Apesar das conquistas e inovações, o blockchain enfrenta desafios cruciais, incluindo escalabilidade, custos de transações, interoperabilidade, privacidade e segurança \cite{Scalability}. A escalabilidade, em particular, emerge como um desafio proeminente, limitando a quantidade de transações que as redes blockchain podem processar, o que impacta sua aplicabilidade em setores de alto volume, como sistemas de pagamento em larga escala \cite{Buterin2013}.

A avaliação da escalabilidade, geralmente medida em Transações por Segundo (TPS), destaca a importância do tamanho dos blocos e dos algoritmos de consenso, como o Proof-of-Work (PoW), na eficiência da rede. Enquanto blocos maiores podem aumentar a capacidade, há o risco de centralização e demanda por mais recursos de armazenamento. Além disso, considerações sobre privacidade e complexidade dos protocolos também influenciam a capacidade de processamento de transações \cite{Proof-of-Work}.

Este trabalho tem como objetivo explorar os limites do Blockchain em relação à escalabilidade e capacidade de processamento de transações em grandes escalas, utilizando métodos sistemáticos de análise de literatura. Apesar de ser uma tecnologia revolucionária com grande potencial para transformar diversos setores, o Blockchain apresenta desafios que podem afetar sua eficiência em larga escala.

A análise crítica concentra-se nos desafios inerentes à escalabilidade e capacidade de processamento em larga escala nas blockchains, examinando aspectos técnicos, como armazenamento, tempos de confirmação, processamento de nós e eficiência energética, enquanto explora questões de governança e adoção generalizada.

Para embasar essa análise crítica, apresentamos uma variedade de perspectivas sobre os desafios de escalabilidade no contexto das blockchains, destacando limitações atuais e explorando soluções propostas pela comunidade acadêmica e pela indústria. Essa abordagem busca informar e orientar futuros desenvolvimentos na busca por soluções mais eficazes e sustentáveis para os desafios enfrentados pela tecnologia blockchain.
			% introdução

\chapter{Introdução}
\label{chapter_intro}


% A tecnologia blockchain tem revolucionado a forma como realizamos transações, assinaturas, contratos, etc., promovendo a descentralização e reduzindo a dependência de aprovações institucionais para validação \cite{Blockchain}. Ela oferece um registro distribuído de transações, compartilhado e verificado por uma rede de computadores  \cite{IBM}. Essas transações são organizadas em blocos, formando uma cadeia de informações imutáveis e protegidas contra adulteração \cite{Immutability}.

% Os primórdios do blockchain remontam à década de 1990, quando Stuart e W. Scott Stornetta introduziram o conceito de "árvore de carimbos de tempo" como uma solução para garantir a integridade de registros digitais. Contudo, foi somente em 2008 que a tecnologia ganhou notoriedade com a publicação do artigo "Bitcoin: A Peer-to-Peer Electronic Cash System" por Satoshi Nakamoto, descrevendo o funcionamento do Bitcoin, a primeira criptomoeda baseada em blockchain \cite{Origem-Blockchain}.

% Apesar de suas realizações e inovações, o blockchain enfrenta desafios significativos, como escalabilidade, custo de transações, interoperabilidade, privacidade e segurança  \cite{Scalability}. A escalabilidade, em particular, é um dos obstáculos mais prementes. A maioria das redes blockchain possui limitações na quantidade de transações processadas em um determinado período, o que pode dificultar sua adoção em áreas que demandam alto volume de transações, como sistemas de pagamento em larga escala \cite{Buterin2013}.

% A escalabilidade é frequentemente avaliada em termos de Transações por Segundo (TPS), uma métrica que avalia a capacidade de uma rede blockchain de processar transações em um segundo. O tamanho dos blocos e os algoritmos de consenso, como o Proof-of-Work (PoW), desempenham um papel crítico na escalabilidade. Blocos maiores podem aumentar a capacidade, mas também podem centralizar a rede e exigir mais recursos de armazenamento. Além disso, a privacidade das transações e os protocolos complexos também afetam a capacidade de processamento de transações \cite{Proof-of-Work}.

% Esta análise crítica se concentra nos desafios da escalabilidade e da capacidade de processamento em larga escala das blockchains. Examina as limitações técnicas, como armazenamento, tempos de confirmação, processamento de nós e eficiência energética . Além disso, discute as questões de governança e adoção em grande escala.

% Para embasar essa análise crítica, apresentamos diversos pontos de vista sobre os desafios da escalabilidade no contexto das blockchains. Destacamos as limitações atuais e exploramos soluções propostas pela comunidade acadêmica e pela indústria .

% A tecnologia blockchain tem redefinido fundamentalmente a maneira como conduzimos transações, assinamos contratos e realizamos uma variedade de operações, eliminando a necessidade de aprovações centralizadas para validação \cite{Blockchain}. Essa inovação estabelece um registro distribuído de transações, verificado por uma rede de computadores e organizado em blocos, criando uma cadeia de informações imutáveis e resistentes à adulteração \cite{IBM, Immutability}.

% Embora as origens do blockchain remontem à década de 1990, quando Stuart e W. Scott Stornetta apresentaram a "árvore de carimbos de tempo" para garantir a integridade de registros digitais, foi apenas em 2008 que a tecnologia ganhou destaque com a publicação do artigo seminal de Satoshi Nakamoto sobre o Bitcoin \cite{Origem-Blockchain}.

% Apesar das conquistas e inovações, o blockchain enfrenta desafios cruciais, incluindo escalabilidade, custos de transações, interoperabilidade, privacidade e segurança \cite{Scalability}. A escalabilidade, em particular, emerge como um desafio proeminente, limitando a quantidade de transações que as redes blockchain podem processar, o que impacta sua aplicabilidade em setores de alto volume, como sistemas de pagamento em larga escala \cite{Buterin2013}.

% A avaliação da escalabilidade, geralmente medida em Transações por Segundo (TPS), destaca a importância do tamanho dos blocos e dos algoritmos de consenso, como o Proof-of-Work (PoW), na eficiência da rede. Enquanto blocos maiores podem aumentar a capacidade, há o risco de centralização e demanda por mais recursos de armazenamento. Além disso, considerações sobre privacidade e complexidade dos protocolos também influenciam a capacidade de processamento de transações \cite{Proof-of-Work}.

% Este trabalho tem como objetivo explorar os limites do Blockchain em relação à escalabilidade e capacidade de processamento de transações em grandes escalas, utilizando métodos sistemáticos de análise de literatura. O Blockchain, apesar de ser uma tecnologia revolucionária com grande potencial para transformar diversos setores, apresenta alguns desafios que podem afetar sua eficiência em larga escala.
% Um dos principais desafios do Blockchain é a escalabilidade, que se refere à capacidade da tecnologia de lidar com um grande número de transações em um curto período de tempo. Isso ocorre porque cada transação é validada por um grande número de nós na rede, o que pode resultar em um gargalo de processamento de dados. Além disso, a capacidade de armazenamento de dados também pode ser um desafio, especialmente em Blockchains públicos, onde todas as transações são armazenadas em cada nó da rede.
% Esta análise crítica concentra-se nos desafios inerentes à escalabilidade e capacidade de processamento em larga escala nas blockchains. Examina aspectos técnicos, como armazenamento, tempos de confirmação, processamento de nós e eficiência energética, enquanto explora questões de governança e adoção generalizada.


% Para embasar essa análise crítica, apresentamos uma variedade de perspectivas sobre os desafios de escalabilidade no contexto das blockchains, destacando limitações atuais e explorando soluções propostas pela comunidade acadêmica e pela indústria. Essa abordagem busca informar e orientar futuros desenvolvimentos na busca por soluções mais eficazes e sustentáveis para os desafios enfrentados pela tecnologia blockchain.

A tecnologia blockchain redefine de maneira fundamental a condução de transações, a assinatura de contratos e a realização de diversas operações, eliminando a necessidade de aprovações centralizadas para validação \cite{Blockchain}. Essa inovação estabelece um registro distribuído de transações, verificado por uma rede de computadores e organizado em blocos, criando uma cadeia de informações imutáveis e resistentes à adulteração \cite{IBM, Immutability}.

Embora as origens do blockchain remontem à década de 1990, quando Stuart e W. Scott Stornetta apresentaram a "árvore de carimbos de tempo" para garantir a integridade de registros digitais, foi apenas em 2008 que a tecnologia ganhou destaque com a publicação do artigo seminal de Satoshi Nakamoto sobre o Bitcoin \cite{Origem-Blockchain}.

Apesar das conquistas e inovações, o blockchain enfrenta desafios cruciais, incluindo escalabilidade, custos de transações, interoperabilidade, privacidade e segurança \cite{Scalability}. A escalabilidade, em particular, emerge como um desafio proeminente, limitando a quantidade de transações que as redes blockchain podem processar, o que impacta sua aplicabilidade em setores de alto volume, como sistemas de pagamento em larga escala \cite{Buterin2013}.

A avaliação da escalabilidade, geralmente medida em Transações por Segundo (TPS), destaca a importância do tamanho dos blocos e dos algoritmos de consenso, como o Proof-of-Work (PoW), na eficiência da rede. Enquanto blocos maiores podem aumentar a capacidade, há o risco de centralização e demanda por mais recursos de armazenamento. Além disso, considerações sobre privacidade e complexidade dos protocolos também influenciam a capacidade de processamento de transações \cite{Proof-of-Work}.

Este trabalho tem como objetivo explorar os limites do Blockchain em relação à escalabilidade e capacidade de processamento de transações em grandes escalas, utilizando métodos sistemáticos de análise de literatura. Apesar de ser uma tecnologia revolucionária com grande potencial para transformar diversos setores, o Blockchain apresenta desafios que podem afetar sua eficiência em larga escala.

A análise crítica concentra-se nos desafios inerentes à escalabilidade e capacidade de processamento em larga escala nas blockchains, examinando aspectos técnicos, como armazenamento, tempos de confirmação, processamento de nós e eficiência energética, enquanto explora questões de governança e adoção generalizada.

Para embasar essa análise crítica, apresentamos uma variedade de perspectivas sobre os desafios de escalabilidade no contexto das blockchains, destacando limitações atuais e explorando soluções propostas pela comunidade acadêmica e pela indústria. Essa abordagem busca informar e orientar futuros desenvolvimentos na busca por soluções mais eficazes e sustentáveis para os desafios enfrentados pela tecnologia blockchain.
   % fondamentos 

\chapter{Introdução}
\label{chapter_intro}


% A tecnologia blockchain tem revolucionado a forma como realizamos transações, assinaturas, contratos, etc., promovendo a descentralização e reduzindo a dependência de aprovações institucionais para validação \cite{Blockchain}. Ela oferece um registro distribuído de transações, compartilhado e verificado por uma rede de computadores  \cite{IBM}. Essas transações são organizadas em blocos, formando uma cadeia de informações imutáveis e protegidas contra adulteração \cite{Immutability}.

% Os primórdios do blockchain remontam à década de 1990, quando Stuart e W. Scott Stornetta introduziram o conceito de "árvore de carimbos de tempo" como uma solução para garantir a integridade de registros digitais. Contudo, foi somente em 2008 que a tecnologia ganhou notoriedade com a publicação do artigo "Bitcoin: A Peer-to-Peer Electronic Cash System" por Satoshi Nakamoto, descrevendo o funcionamento do Bitcoin, a primeira criptomoeda baseada em blockchain \cite{Origem-Blockchain}.

% Apesar de suas realizações e inovações, o blockchain enfrenta desafios significativos, como escalabilidade, custo de transações, interoperabilidade, privacidade e segurança  \cite{Scalability}. A escalabilidade, em particular, é um dos obstáculos mais prementes. A maioria das redes blockchain possui limitações na quantidade de transações processadas em um determinado período, o que pode dificultar sua adoção em áreas que demandam alto volume de transações, como sistemas de pagamento em larga escala \cite{Buterin2013}.

% A escalabilidade é frequentemente avaliada em termos de Transações por Segundo (TPS), uma métrica que avalia a capacidade de uma rede blockchain de processar transações em um segundo. O tamanho dos blocos e os algoritmos de consenso, como o Proof-of-Work (PoW), desempenham um papel crítico na escalabilidade. Blocos maiores podem aumentar a capacidade, mas também podem centralizar a rede e exigir mais recursos de armazenamento. Além disso, a privacidade das transações e os protocolos complexos também afetam a capacidade de processamento de transações \cite{Proof-of-Work}.

% Esta análise crítica se concentra nos desafios da escalabilidade e da capacidade de processamento em larga escala das blockchains. Examina as limitações técnicas, como armazenamento, tempos de confirmação, processamento de nós e eficiência energética . Além disso, discute as questões de governança e adoção em grande escala.

% Para embasar essa análise crítica, apresentamos diversos pontos de vista sobre os desafios da escalabilidade no contexto das blockchains. Destacamos as limitações atuais e exploramos soluções propostas pela comunidade acadêmica e pela indústria .

% A tecnologia blockchain tem redefinido fundamentalmente a maneira como conduzimos transações, assinamos contratos e realizamos uma variedade de operações, eliminando a necessidade de aprovações centralizadas para validação \cite{Blockchain}. Essa inovação estabelece um registro distribuído de transações, verificado por uma rede de computadores e organizado em blocos, criando uma cadeia de informações imutáveis e resistentes à adulteração \cite{IBM, Immutability}.

% Embora as origens do blockchain remontem à década de 1990, quando Stuart e W. Scott Stornetta apresentaram a "árvore de carimbos de tempo" para garantir a integridade de registros digitais, foi apenas em 2008 que a tecnologia ganhou destaque com a publicação do artigo seminal de Satoshi Nakamoto sobre o Bitcoin \cite{Origem-Blockchain}.

% Apesar das conquistas e inovações, o blockchain enfrenta desafios cruciais, incluindo escalabilidade, custos de transações, interoperabilidade, privacidade e segurança \cite{Scalability}. A escalabilidade, em particular, emerge como um desafio proeminente, limitando a quantidade de transações que as redes blockchain podem processar, o que impacta sua aplicabilidade em setores de alto volume, como sistemas de pagamento em larga escala \cite{Buterin2013}.

% A avaliação da escalabilidade, geralmente medida em Transações por Segundo (TPS), destaca a importância do tamanho dos blocos e dos algoritmos de consenso, como o Proof-of-Work (PoW), na eficiência da rede. Enquanto blocos maiores podem aumentar a capacidade, há o risco de centralização e demanda por mais recursos de armazenamento. Além disso, considerações sobre privacidade e complexidade dos protocolos também influenciam a capacidade de processamento de transações \cite{Proof-of-Work}.

% Este trabalho tem como objetivo explorar os limites do Blockchain em relação à escalabilidade e capacidade de processamento de transações em grandes escalas, utilizando métodos sistemáticos de análise de literatura. O Blockchain, apesar de ser uma tecnologia revolucionária com grande potencial para transformar diversos setores, apresenta alguns desafios que podem afetar sua eficiência em larga escala.
% Um dos principais desafios do Blockchain é a escalabilidade, que se refere à capacidade da tecnologia de lidar com um grande número de transações em um curto período de tempo. Isso ocorre porque cada transação é validada por um grande número de nós na rede, o que pode resultar em um gargalo de processamento de dados. Além disso, a capacidade de armazenamento de dados também pode ser um desafio, especialmente em Blockchains públicos, onde todas as transações são armazenadas em cada nó da rede.
% Esta análise crítica concentra-se nos desafios inerentes à escalabilidade e capacidade de processamento em larga escala nas blockchains. Examina aspectos técnicos, como armazenamento, tempos de confirmação, processamento de nós e eficiência energética, enquanto explora questões de governança e adoção generalizada.


% Para embasar essa análise crítica, apresentamos uma variedade de perspectivas sobre os desafios de escalabilidade no contexto das blockchains, destacando limitações atuais e explorando soluções propostas pela comunidade acadêmica e pela indústria. Essa abordagem busca informar e orientar futuros desenvolvimentos na busca por soluções mais eficazes e sustentáveis para os desafios enfrentados pela tecnologia blockchain.

A tecnologia blockchain redefine de maneira fundamental a condução de transações, a assinatura de contratos e a realização de diversas operações, eliminando a necessidade de aprovações centralizadas para validação \cite{Blockchain}. Essa inovação estabelece um registro distribuído de transações, verificado por uma rede de computadores e organizado em blocos, criando uma cadeia de informações imutáveis e resistentes à adulteração \cite{IBM, Immutability}.

Embora as origens do blockchain remontem à década de 1990, quando Stuart e W. Scott Stornetta apresentaram a "árvore de carimbos de tempo" para garantir a integridade de registros digitais, foi apenas em 2008 que a tecnologia ganhou destaque com a publicação do artigo seminal de Satoshi Nakamoto sobre o Bitcoin \cite{Origem-Blockchain}.

Apesar das conquistas e inovações, o blockchain enfrenta desafios cruciais, incluindo escalabilidade, custos de transações, interoperabilidade, privacidade e segurança \cite{Scalability}. A escalabilidade, em particular, emerge como um desafio proeminente, limitando a quantidade de transações que as redes blockchain podem processar, o que impacta sua aplicabilidade em setores de alto volume, como sistemas de pagamento em larga escala \cite{Buterin2013}.

A avaliação da escalabilidade, geralmente medida em Transações por Segundo (TPS), destaca a importância do tamanho dos blocos e dos algoritmos de consenso, como o Proof-of-Work (PoW), na eficiência da rede. Enquanto blocos maiores podem aumentar a capacidade, há o risco de centralização e demanda por mais recursos de armazenamento. Além disso, considerações sobre privacidade e complexidade dos protocolos também influenciam a capacidade de processamento de transações \cite{Proof-of-Work}.

Este trabalho tem como objetivo explorar os limites do Blockchain em relação à escalabilidade e capacidade de processamento de transações em grandes escalas, utilizando métodos sistemáticos de análise de literatura. Apesar de ser uma tecnologia revolucionária com grande potencial para transformar diversos setores, o Blockchain apresenta desafios que podem afetar sua eficiência em larga escala.

A análise crítica concentra-se nos desafios inerentes à escalabilidade e capacidade de processamento em larga escala nas blockchains, examinando aspectos técnicos, como armazenamento, tempos de confirmação, processamento de nós e eficiência energética, enquanto explora questões de governança e adoção generalizada.

Para embasar essa análise crítica, apresentamos uma variedade de perspectivas sobre os desafios de escalabilidade no contexto das blockchains, destacando limitações atuais e explorando soluções propostas pela comunidade acadêmica e pela indústria. Essa abordagem busca informar e orientar futuros desenvolvimentos na busca por soluções mais eficazes e sustentáveis para os desafios enfrentados pela tecnologia blockchain.
 % objetivo

\chapter{Introdução}
\label{chapter_intro}


% A tecnologia blockchain tem revolucionado a forma como realizamos transações, assinaturas, contratos, etc., promovendo a descentralização e reduzindo a dependência de aprovações institucionais para validação \cite{Blockchain}. Ela oferece um registro distribuído de transações, compartilhado e verificado por uma rede de computadores  \cite{IBM}. Essas transações são organizadas em blocos, formando uma cadeia de informações imutáveis e protegidas contra adulteração \cite{Immutability}.

% Os primórdios do blockchain remontam à década de 1990, quando Stuart e W. Scott Stornetta introduziram o conceito de "árvore de carimbos de tempo" como uma solução para garantir a integridade de registros digitais. Contudo, foi somente em 2008 que a tecnologia ganhou notoriedade com a publicação do artigo "Bitcoin: A Peer-to-Peer Electronic Cash System" por Satoshi Nakamoto, descrevendo o funcionamento do Bitcoin, a primeira criptomoeda baseada em blockchain \cite{Origem-Blockchain}.

% Apesar de suas realizações e inovações, o blockchain enfrenta desafios significativos, como escalabilidade, custo de transações, interoperabilidade, privacidade e segurança  \cite{Scalability}. A escalabilidade, em particular, é um dos obstáculos mais prementes. A maioria das redes blockchain possui limitações na quantidade de transações processadas em um determinado período, o que pode dificultar sua adoção em áreas que demandam alto volume de transações, como sistemas de pagamento em larga escala \cite{Buterin2013}.

% A escalabilidade é frequentemente avaliada em termos de Transações por Segundo (TPS), uma métrica que avalia a capacidade de uma rede blockchain de processar transações em um segundo. O tamanho dos blocos e os algoritmos de consenso, como o Proof-of-Work (PoW), desempenham um papel crítico na escalabilidade. Blocos maiores podem aumentar a capacidade, mas também podem centralizar a rede e exigir mais recursos de armazenamento. Além disso, a privacidade das transações e os protocolos complexos também afetam a capacidade de processamento de transações \cite{Proof-of-Work}.

% Esta análise crítica se concentra nos desafios da escalabilidade e da capacidade de processamento em larga escala das blockchains. Examina as limitações técnicas, como armazenamento, tempos de confirmação, processamento de nós e eficiência energética . Além disso, discute as questões de governança e adoção em grande escala.

% Para embasar essa análise crítica, apresentamos diversos pontos de vista sobre os desafios da escalabilidade no contexto das blockchains. Destacamos as limitações atuais e exploramos soluções propostas pela comunidade acadêmica e pela indústria .

% A tecnologia blockchain tem redefinido fundamentalmente a maneira como conduzimos transações, assinamos contratos e realizamos uma variedade de operações, eliminando a necessidade de aprovações centralizadas para validação \cite{Blockchain}. Essa inovação estabelece um registro distribuído de transações, verificado por uma rede de computadores e organizado em blocos, criando uma cadeia de informações imutáveis e resistentes à adulteração \cite{IBM, Immutability}.

% Embora as origens do blockchain remontem à década de 1990, quando Stuart e W. Scott Stornetta apresentaram a "árvore de carimbos de tempo" para garantir a integridade de registros digitais, foi apenas em 2008 que a tecnologia ganhou destaque com a publicação do artigo seminal de Satoshi Nakamoto sobre o Bitcoin \cite{Origem-Blockchain}.

% Apesar das conquistas e inovações, o blockchain enfrenta desafios cruciais, incluindo escalabilidade, custos de transações, interoperabilidade, privacidade e segurança \cite{Scalability}. A escalabilidade, em particular, emerge como um desafio proeminente, limitando a quantidade de transações que as redes blockchain podem processar, o que impacta sua aplicabilidade em setores de alto volume, como sistemas de pagamento em larga escala \cite{Buterin2013}.

% A avaliação da escalabilidade, geralmente medida em Transações por Segundo (TPS), destaca a importância do tamanho dos blocos e dos algoritmos de consenso, como o Proof-of-Work (PoW), na eficiência da rede. Enquanto blocos maiores podem aumentar a capacidade, há o risco de centralização e demanda por mais recursos de armazenamento. Além disso, considerações sobre privacidade e complexidade dos protocolos também influenciam a capacidade de processamento de transações \cite{Proof-of-Work}.

% Este trabalho tem como objetivo explorar os limites do Blockchain em relação à escalabilidade e capacidade de processamento de transações em grandes escalas, utilizando métodos sistemáticos de análise de literatura. O Blockchain, apesar de ser uma tecnologia revolucionária com grande potencial para transformar diversos setores, apresenta alguns desafios que podem afetar sua eficiência em larga escala.
% Um dos principais desafios do Blockchain é a escalabilidade, que se refere à capacidade da tecnologia de lidar com um grande número de transações em um curto período de tempo. Isso ocorre porque cada transação é validada por um grande número de nós na rede, o que pode resultar em um gargalo de processamento de dados. Além disso, a capacidade de armazenamento de dados também pode ser um desafio, especialmente em Blockchains públicos, onde todas as transações são armazenadas em cada nó da rede.
% Esta análise crítica concentra-se nos desafios inerentes à escalabilidade e capacidade de processamento em larga escala nas blockchains. Examina aspectos técnicos, como armazenamento, tempos de confirmação, processamento de nós e eficiência energética, enquanto explora questões de governança e adoção generalizada.


% Para embasar essa análise crítica, apresentamos uma variedade de perspectivas sobre os desafios de escalabilidade no contexto das blockchains, destacando limitações atuais e explorando soluções propostas pela comunidade acadêmica e pela indústria. Essa abordagem busca informar e orientar futuros desenvolvimentos na busca por soluções mais eficazes e sustentáveis para os desafios enfrentados pela tecnologia blockchain.

A tecnologia blockchain redefine de maneira fundamental a condução de transações, a assinatura de contratos e a realização de diversas operações, eliminando a necessidade de aprovações centralizadas para validação \cite{Blockchain}. Essa inovação estabelece um registro distribuído de transações, verificado por uma rede de computadores e organizado em blocos, criando uma cadeia de informações imutáveis e resistentes à adulteração \cite{IBM, Immutability}.

Embora as origens do blockchain remontem à década de 1990, quando Stuart e W. Scott Stornetta apresentaram a "árvore de carimbos de tempo" para garantir a integridade de registros digitais, foi apenas em 2008 que a tecnologia ganhou destaque com a publicação do artigo seminal de Satoshi Nakamoto sobre o Bitcoin \cite{Origem-Blockchain}.

Apesar das conquistas e inovações, o blockchain enfrenta desafios cruciais, incluindo escalabilidade, custos de transações, interoperabilidade, privacidade e segurança \cite{Scalability}. A escalabilidade, em particular, emerge como um desafio proeminente, limitando a quantidade de transações que as redes blockchain podem processar, o que impacta sua aplicabilidade em setores de alto volume, como sistemas de pagamento em larga escala \cite{Buterin2013}.

A avaliação da escalabilidade, geralmente medida em Transações por Segundo (TPS), destaca a importância do tamanho dos blocos e dos algoritmos de consenso, como o Proof-of-Work (PoW), na eficiência da rede. Enquanto blocos maiores podem aumentar a capacidade, há o risco de centralização e demanda por mais recursos de armazenamento. Além disso, considerações sobre privacidade e complexidade dos protocolos também influenciam a capacidade de processamento de transações \cite{Proof-of-Work}.

Este trabalho tem como objetivo explorar os limites do Blockchain em relação à escalabilidade e capacidade de processamento de transações em grandes escalas, utilizando métodos sistemáticos de análise de literatura. Apesar de ser uma tecnologia revolucionária com grande potencial para transformar diversos setores, o Blockchain apresenta desafios que podem afetar sua eficiência em larga escala.

A análise crítica concentra-se nos desafios inerentes à escalabilidade e capacidade de processamento em larga escala nas blockchains, examinando aspectos técnicos, como armazenamento, tempos de confirmação, processamento de nós e eficiência energética, enquanto explora questões de governança e adoção generalizada.

Para embasar essa análise crítica, apresentamos uma variedade de perspectivas sobre os desafios de escalabilidade no contexto das blockchains, destacando limitações atuais e explorando soluções propostas pela comunidade acadêmica e pela indústria. Essa abordagem busca informar e orientar futuros desenvolvimentos na busca por soluções mais eficazes e sustentáveis para os desafios enfrentados pela tecnologia blockchain.
 % analise de desempenho

\chapter{Introdução}
\label{chapter_intro}


% A tecnologia blockchain tem revolucionado a forma como realizamos transações, assinaturas, contratos, etc., promovendo a descentralização e reduzindo a dependência de aprovações institucionais para validação \cite{Blockchain}. Ela oferece um registro distribuído de transações, compartilhado e verificado por uma rede de computadores  \cite{IBM}. Essas transações são organizadas em blocos, formando uma cadeia de informações imutáveis e protegidas contra adulteração \cite{Immutability}.

% Os primórdios do blockchain remontam à década de 1990, quando Stuart e W. Scott Stornetta introduziram o conceito de "árvore de carimbos de tempo" como uma solução para garantir a integridade de registros digitais. Contudo, foi somente em 2008 que a tecnologia ganhou notoriedade com a publicação do artigo "Bitcoin: A Peer-to-Peer Electronic Cash System" por Satoshi Nakamoto, descrevendo o funcionamento do Bitcoin, a primeira criptomoeda baseada em blockchain \cite{Origem-Blockchain}.

% Apesar de suas realizações e inovações, o blockchain enfrenta desafios significativos, como escalabilidade, custo de transações, interoperabilidade, privacidade e segurança  \cite{Scalability}. A escalabilidade, em particular, é um dos obstáculos mais prementes. A maioria das redes blockchain possui limitações na quantidade de transações processadas em um determinado período, o que pode dificultar sua adoção em áreas que demandam alto volume de transações, como sistemas de pagamento em larga escala \cite{Buterin2013}.

% A escalabilidade é frequentemente avaliada em termos de Transações por Segundo (TPS), uma métrica que avalia a capacidade de uma rede blockchain de processar transações em um segundo. O tamanho dos blocos e os algoritmos de consenso, como o Proof-of-Work (PoW), desempenham um papel crítico na escalabilidade. Blocos maiores podem aumentar a capacidade, mas também podem centralizar a rede e exigir mais recursos de armazenamento. Além disso, a privacidade das transações e os protocolos complexos também afetam a capacidade de processamento de transações \cite{Proof-of-Work}.

% Esta análise crítica se concentra nos desafios da escalabilidade e da capacidade de processamento em larga escala das blockchains. Examina as limitações técnicas, como armazenamento, tempos de confirmação, processamento de nós e eficiência energética . Além disso, discute as questões de governança e adoção em grande escala.

% Para embasar essa análise crítica, apresentamos diversos pontos de vista sobre os desafios da escalabilidade no contexto das blockchains. Destacamos as limitações atuais e exploramos soluções propostas pela comunidade acadêmica e pela indústria .

% A tecnologia blockchain tem redefinido fundamentalmente a maneira como conduzimos transações, assinamos contratos e realizamos uma variedade de operações, eliminando a necessidade de aprovações centralizadas para validação \cite{Blockchain}. Essa inovação estabelece um registro distribuído de transações, verificado por uma rede de computadores e organizado em blocos, criando uma cadeia de informações imutáveis e resistentes à adulteração \cite{IBM, Immutability}.

% Embora as origens do blockchain remontem à década de 1990, quando Stuart e W. Scott Stornetta apresentaram a "árvore de carimbos de tempo" para garantir a integridade de registros digitais, foi apenas em 2008 que a tecnologia ganhou destaque com a publicação do artigo seminal de Satoshi Nakamoto sobre o Bitcoin \cite{Origem-Blockchain}.

% Apesar das conquistas e inovações, o blockchain enfrenta desafios cruciais, incluindo escalabilidade, custos de transações, interoperabilidade, privacidade e segurança \cite{Scalability}. A escalabilidade, em particular, emerge como um desafio proeminente, limitando a quantidade de transações que as redes blockchain podem processar, o que impacta sua aplicabilidade em setores de alto volume, como sistemas de pagamento em larga escala \cite{Buterin2013}.

% A avaliação da escalabilidade, geralmente medida em Transações por Segundo (TPS), destaca a importância do tamanho dos blocos e dos algoritmos de consenso, como o Proof-of-Work (PoW), na eficiência da rede. Enquanto blocos maiores podem aumentar a capacidade, há o risco de centralização e demanda por mais recursos de armazenamento. Além disso, considerações sobre privacidade e complexidade dos protocolos também influenciam a capacidade de processamento de transações \cite{Proof-of-Work}.

% Este trabalho tem como objetivo explorar os limites do Blockchain em relação à escalabilidade e capacidade de processamento de transações em grandes escalas, utilizando métodos sistemáticos de análise de literatura. O Blockchain, apesar de ser uma tecnologia revolucionária com grande potencial para transformar diversos setores, apresenta alguns desafios que podem afetar sua eficiência em larga escala.
% Um dos principais desafios do Blockchain é a escalabilidade, que se refere à capacidade da tecnologia de lidar com um grande número de transações em um curto período de tempo. Isso ocorre porque cada transação é validada por um grande número de nós na rede, o que pode resultar em um gargalo de processamento de dados. Além disso, a capacidade de armazenamento de dados também pode ser um desafio, especialmente em Blockchains públicos, onde todas as transações são armazenadas em cada nó da rede.
% Esta análise crítica concentra-se nos desafios inerentes à escalabilidade e capacidade de processamento em larga escala nas blockchains. Examina aspectos técnicos, como armazenamento, tempos de confirmação, processamento de nós e eficiência energética, enquanto explora questões de governança e adoção generalizada.


% Para embasar essa análise crítica, apresentamos uma variedade de perspectivas sobre os desafios de escalabilidade no contexto das blockchains, destacando limitações atuais e explorando soluções propostas pela comunidade acadêmica e pela indústria. Essa abordagem busca informar e orientar futuros desenvolvimentos na busca por soluções mais eficazes e sustentáveis para os desafios enfrentados pela tecnologia blockchain.

A tecnologia blockchain redefine de maneira fundamental a condução de transações, a assinatura de contratos e a realização de diversas operações, eliminando a necessidade de aprovações centralizadas para validação \cite{Blockchain}. Essa inovação estabelece um registro distribuído de transações, verificado por uma rede de computadores e organizado em blocos, criando uma cadeia de informações imutáveis e resistentes à adulteração \cite{IBM, Immutability}.

Embora as origens do blockchain remontem à década de 1990, quando Stuart e W. Scott Stornetta apresentaram a "árvore de carimbos de tempo" para garantir a integridade de registros digitais, foi apenas em 2008 que a tecnologia ganhou destaque com a publicação do artigo seminal de Satoshi Nakamoto sobre o Bitcoin \cite{Origem-Blockchain}.

Apesar das conquistas e inovações, o blockchain enfrenta desafios cruciais, incluindo escalabilidade, custos de transações, interoperabilidade, privacidade e segurança \cite{Scalability}. A escalabilidade, em particular, emerge como um desafio proeminente, limitando a quantidade de transações que as redes blockchain podem processar, o que impacta sua aplicabilidade em setores de alto volume, como sistemas de pagamento em larga escala \cite{Buterin2013}.

A avaliação da escalabilidade, geralmente medida em Transações por Segundo (TPS), destaca a importância do tamanho dos blocos e dos algoritmos de consenso, como o Proof-of-Work (PoW), na eficiência da rede. Enquanto blocos maiores podem aumentar a capacidade, há o risco de centralização e demanda por mais recursos de armazenamento. Além disso, considerações sobre privacidade e complexidade dos protocolos também influenciam a capacidade de processamento de transações \cite{Proof-of-Work}.

Este trabalho tem como objetivo explorar os limites do Blockchain em relação à escalabilidade e capacidade de processamento de transações em grandes escalas, utilizando métodos sistemáticos de análise de literatura. Apesar de ser uma tecnologia revolucionária com grande potencial para transformar diversos setores, o Blockchain apresenta desafios que podem afetar sua eficiência em larga escala.

A análise crítica concentra-se nos desafios inerentes à escalabilidade e capacidade de processamento em larga escala nas blockchains, examinando aspectos técnicos, como armazenamento, tempos de confirmação, processamento de nós e eficiência energética, enquanto explora questões de governança e adoção generalizada.

Para embasar essa análise crítica, apresentamos uma variedade de perspectivas sobre os desafios de escalabilidade no contexto das blockchains, destacando limitações atuais e explorando soluções propostas pela comunidade acadêmica e pela indústria. Essa abordagem busca informar e orientar futuros desenvolvimentos na busca por soluções mais eficazes e sustentáveis para os desafios enfrentados pela tecnologia blockchain.
   % experimentos

\chapter{Introdução}
\label{chapter_intro}


% A tecnologia blockchain tem revolucionado a forma como realizamos transações, assinaturas, contratos, etc., promovendo a descentralização e reduzindo a dependência de aprovações institucionais para validação \cite{Blockchain}. Ela oferece um registro distribuído de transações, compartilhado e verificado por uma rede de computadores  \cite{IBM}. Essas transações são organizadas em blocos, formando uma cadeia de informações imutáveis e protegidas contra adulteração \cite{Immutability}.

% Os primórdios do blockchain remontam à década de 1990, quando Stuart e W. Scott Stornetta introduziram o conceito de "árvore de carimbos de tempo" como uma solução para garantir a integridade de registros digitais. Contudo, foi somente em 2008 que a tecnologia ganhou notoriedade com a publicação do artigo "Bitcoin: A Peer-to-Peer Electronic Cash System" por Satoshi Nakamoto, descrevendo o funcionamento do Bitcoin, a primeira criptomoeda baseada em blockchain \cite{Origem-Blockchain}.

% Apesar de suas realizações e inovações, o blockchain enfrenta desafios significativos, como escalabilidade, custo de transações, interoperabilidade, privacidade e segurança  \cite{Scalability}. A escalabilidade, em particular, é um dos obstáculos mais prementes. A maioria das redes blockchain possui limitações na quantidade de transações processadas em um determinado período, o que pode dificultar sua adoção em áreas que demandam alto volume de transações, como sistemas de pagamento em larga escala \cite{Buterin2013}.

% A escalabilidade é frequentemente avaliada em termos de Transações por Segundo (TPS), uma métrica que avalia a capacidade de uma rede blockchain de processar transações em um segundo. O tamanho dos blocos e os algoritmos de consenso, como o Proof-of-Work (PoW), desempenham um papel crítico na escalabilidade. Blocos maiores podem aumentar a capacidade, mas também podem centralizar a rede e exigir mais recursos de armazenamento. Além disso, a privacidade das transações e os protocolos complexos também afetam a capacidade de processamento de transações \cite{Proof-of-Work}.

% Esta análise crítica se concentra nos desafios da escalabilidade e da capacidade de processamento em larga escala das blockchains. Examina as limitações técnicas, como armazenamento, tempos de confirmação, processamento de nós e eficiência energética . Além disso, discute as questões de governança e adoção em grande escala.

% Para embasar essa análise crítica, apresentamos diversos pontos de vista sobre os desafios da escalabilidade no contexto das blockchains. Destacamos as limitações atuais e exploramos soluções propostas pela comunidade acadêmica e pela indústria .

% A tecnologia blockchain tem redefinido fundamentalmente a maneira como conduzimos transações, assinamos contratos e realizamos uma variedade de operações, eliminando a necessidade de aprovações centralizadas para validação \cite{Blockchain}. Essa inovação estabelece um registro distribuído de transações, verificado por uma rede de computadores e organizado em blocos, criando uma cadeia de informações imutáveis e resistentes à adulteração \cite{IBM, Immutability}.

% Embora as origens do blockchain remontem à década de 1990, quando Stuart e W. Scott Stornetta apresentaram a "árvore de carimbos de tempo" para garantir a integridade de registros digitais, foi apenas em 2008 que a tecnologia ganhou destaque com a publicação do artigo seminal de Satoshi Nakamoto sobre o Bitcoin \cite{Origem-Blockchain}.

% Apesar das conquistas e inovações, o blockchain enfrenta desafios cruciais, incluindo escalabilidade, custos de transações, interoperabilidade, privacidade e segurança \cite{Scalability}. A escalabilidade, em particular, emerge como um desafio proeminente, limitando a quantidade de transações que as redes blockchain podem processar, o que impacta sua aplicabilidade em setores de alto volume, como sistemas de pagamento em larga escala \cite{Buterin2013}.

% A avaliação da escalabilidade, geralmente medida em Transações por Segundo (TPS), destaca a importância do tamanho dos blocos e dos algoritmos de consenso, como o Proof-of-Work (PoW), na eficiência da rede. Enquanto blocos maiores podem aumentar a capacidade, há o risco de centralização e demanda por mais recursos de armazenamento. Além disso, considerações sobre privacidade e complexidade dos protocolos também influenciam a capacidade de processamento de transações \cite{Proof-of-Work}.

% Este trabalho tem como objetivo explorar os limites do Blockchain em relação à escalabilidade e capacidade de processamento de transações em grandes escalas, utilizando métodos sistemáticos de análise de literatura. O Blockchain, apesar de ser uma tecnologia revolucionária com grande potencial para transformar diversos setores, apresenta alguns desafios que podem afetar sua eficiência em larga escala.
% Um dos principais desafios do Blockchain é a escalabilidade, que se refere à capacidade da tecnologia de lidar com um grande número de transações em um curto período de tempo. Isso ocorre porque cada transação é validada por um grande número de nós na rede, o que pode resultar em um gargalo de processamento de dados. Além disso, a capacidade de armazenamento de dados também pode ser um desafio, especialmente em Blockchains públicos, onde todas as transações são armazenadas em cada nó da rede.
% Esta análise crítica concentra-se nos desafios inerentes à escalabilidade e capacidade de processamento em larga escala nas blockchains. Examina aspectos técnicos, como armazenamento, tempos de confirmação, processamento de nós e eficiência energética, enquanto explora questões de governança e adoção generalizada.


% Para embasar essa análise crítica, apresentamos uma variedade de perspectivas sobre os desafios de escalabilidade no contexto das blockchains, destacando limitações atuais e explorando soluções propostas pela comunidade acadêmica e pela indústria. Essa abordagem busca informar e orientar futuros desenvolvimentos na busca por soluções mais eficazes e sustentáveis para os desafios enfrentados pela tecnologia blockchain.

A tecnologia blockchain redefine de maneira fundamental a condução de transações, a assinatura de contratos e a realização de diversas operações, eliminando a necessidade de aprovações centralizadas para validação \cite{Blockchain}. Essa inovação estabelece um registro distribuído de transações, verificado por uma rede de computadores e organizado em blocos, criando uma cadeia de informações imutáveis e resistentes à adulteração \cite{IBM, Immutability}.

Embora as origens do blockchain remontem à década de 1990, quando Stuart e W. Scott Stornetta apresentaram a "árvore de carimbos de tempo" para garantir a integridade de registros digitais, foi apenas em 2008 que a tecnologia ganhou destaque com a publicação do artigo seminal de Satoshi Nakamoto sobre o Bitcoin \cite{Origem-Blockchain}.

Apesar das conquistas e inovações, o blockchain enfrenta desafios cruciais, incluindo escalabilidade, custos de transações, interoperabilidade, privacidade e segurança \cite{Scalability}. A escalabilidade, em particular, emerge como um desafio proeminente, limitando a quantidade de transações que as redes blockchain podem processar, o que impacta sua aplicabilidade em setores de alto volume, como sistemas de pagamento em larga escala \cite{Buterin2013}.

A avaliação da escalabilidade, geralmente medida em Transações por Segundo (TPS), destaca a importância do tamanho dos blocos e dos algoritmos de consenso, como o Proof-of-Work (PoW), na eficiência da rede. Enquanto blocos maiores podem aumentar a capacidade, há o risco de centralização e demanda por mais recursos de armazenamento. Além disso, considerações sobre privacidade e complexidade dos protocolos também influenciam a capacidade de processamento de transações \cite{Proof-of-Work}.

Este trabalho tem como objetivo explorar os limites do Blockchain em relação à escalabilidade e capacidade de processamento de transações em grandes escalas, utilizando métodos sistemáticos de análise de literatura. Apesar de ser uma tecnologia revolucionária com grande potencial para transformar diversos setores, o Blockchain apresenta desafios que podem afetar sua eficiência em larga escala.

A análise crítica concentra-se nos desafios inerentes à escalabilidade e capacidade de processamento em larga escala nas blockchains, examinando aspectos técnicos, como armazenamento, tempos de confirmação, processamento de nós e eficiência energética, enquanto explora questões de governança e adoção generalizada.

Para embasar essa análise crítica, apresentamos uma variedade de perspectivas sobre os desafios de escalabilidade no contexto das blockchains, destacando limitações atuais e explorando soluções propostas pela comunidade acadêmica e pela indústria. Essa abordagem busca informar e orientar futuros desenvolvimentos na busca por soluções mais eficazes e sustentáveis para os desafios enfrentados pela tecnologia blockchain.


\chapter{Introdução}
\label{chapter_intro}


% A tecnologia blockchain tem revolucionado a forma como realizamos transações, assinaturas, contratos, etc., promovendo a descentralização e reduzindo a dependência de aprovações institucionais para validação \cite{Blockchain}. Ela oferece um registro distribuído de transações, compartilhado e verificado por uma rede de computadores  \cite{IBM}. Essas transações são organizadas em blocos, formando uma cadeia de informações imutáveis e protegidas contra adulteração \cite{Immutability}.

% Os primórdios do blockchain remontam à década de 1990, quando Stuart e W. Scott Stornetta introduziram o conceito de "árvore de carimbos de tempo" como uma solução para garantir a integridade de registros digitais. Contudo, foi somente em 2008 que a tecnologia ganhou notoriedade com a publicação do artigo "Bitcoin: A Peer-to-Peer Electronic Cash System" por Satoshi Nakamoto, descrevendo o funcionamento do Bitcoin, a primeira criptomoeda baseada em blockchain \cite{Origem-Blockchain}.

% Apesar de suas realizações e inovações, o blockchain enfrenta desafios significativos, como escalabilidade, custo de transações, interoperabilidade, privacidade e segurança  \cite{Scalability}. A escalabilidade, em particular, é um dos obstáculos mais prementes. A maioria das redes blockchain possui limitações na quantidade de transações processadas em um determinado período, o que pode dificultar sua adoção em áreas que demandam alto volume de transações, como sistemas de pagamento em larga escala \cite{Buterin2013}.

% A escalabilidade é frequentemente avaliada em termos de Transações por Segundo (TPS), uma métrica que avalia a capacidade de uma rede blockchain de processar transações em um segundo. O tamanho dos blocos e os algoritmos de consenso, como o Proof-of-Work (PoW), desempenham um papel crítico na escalabilidade. Blocos maiores podem aumentar a capacidade, mas também podem centralizar a rede e exigir mais recursos de armazenamento. Além disso, a privacidade das transações e os protocolos complexos também afetam a capacidade de processamento de transações \cite{Proof-of-Work}.

% Esta análise crítica se concentra nos desafios da escalabilidade e da capacidade de processamento em larga escala das blockchains. Examina as limitações técnicas, como armazenamento, tempos de confirmação, processamento de nós e eficiência energética . Além disso, discute as questões de governança e adoção em grande escala.

% Para embasar essa análise crítica, apresentamos diversos pontos de vista sobre os desafios da escalabilidade no contexto das blockchains. Destacamos as limitações atuais e exploramos soluções propostas pela comunidade acadêmica e pela indústria .

% A tecnologia blockchain tem redefinido fundamentalmente a maneira como conduzimos transações, assinamos contratos e realizamos uma variedade de operações, eliminando a necessidade de aprovações centralizadas para validação \cite{Blockchain}. Essa inovação estabelece um registro distribuído de transações, verificado por uma rede de computadores e organizado em blocos, criando uma cadeia de informações imutáveis e resistentes à adulteração \cite{IBM, Immutability}.

% Embora as origens do blockchain remontem à década de 1990, quando Stuart e W. Scott Stornetta apresentaram a "árvore de carimbos de tempo" para garantir a integridade de registros digitais, foi apenas em 2008 que a tecnologia ganhou destaque com a publicação do artigo seminal de Satoshi Nakamoto sobre o Bitcoin \cite{Origem-Blockchain}.

% Apesar das conquistas e inovações, o blockchain enfrenta desafios cruciais, incluindo escalabilidade, custos de transações, interoperabilidade, privacidade e segurança \cite{Scalability}. A escalabilidade, em particular, emerge como um desafio proeminente, limitando a quantidade de transações que as redes blockchain podem processar, o que impacta sua aplicabilidade em setores de alto volume, como sistemas de pagamento em larga escala \cite{Buterin2013}.

% A avaliação da escalabilidade, geralmente medida em Transações por Segundo (TPS), destaca a importância do tamanho dos blocos e dos algoritmos de consenso, como o Proof-of-Work (PoW), na eficiência da rede. Enquanto blocos maiores podem aumentar a capacidade, há o risco de centralização e demanda por mais recursos de armazenamento. Além disso, considerações sobre privacidade e complexidade dos protocolos também influenciam a capacidade de processamento de transações \cite{Proof-of-Work}.

% Este trabalho tem como objetivo explorar os limites do Blockchain em relação à escalabilidade e capacidade de processamento de transações em grandes escalas, utilizando métodos sistemáticos de análise de literatura. O Blockchain, apesar de ser uma tecnologia revolucionária com grande potencial para transformar diversos setores, apresenta alguns desafios que podem afetar sua eficiência em larga escala.
% Um dos principais desafios do Blockchain é a escalabilidade, que se refere à capacidade da tecnologia de lidar com um grande número de transações em um curto período de tempo. Isso ocorre porque cada transação é validada por um grande número de nós na rede, o que pode resultar em um gargalo de processamento de dados. Além disso, a capacidade de armazenamento de dados também pode ser um desafio, especialmente em Blockchains públicos, onde todas as transações são armazenadas em cada nó da rede.
% Esta análise crítica concentra-se nos desafios inerentes à escalabilidade e capacidade de processamento em larga escala nas blockchains. Examina aspectos técnicos, como armazenamento, tempos de confirmação, processamento de nós e eficiência energética, enquanto explora questões de governança e adoção generalizada.


% Para embasar essa análise crítica, apresentamos uma variedade de perspectivas sobre os desafios de escalabilidade no contexto das blockchains, destacando limitações atuais e explorando soluções propostas pela comunidade acadêmica e pela indústria. Essa abordagem busca informar e orientar futuros desenvolvimentos na busca por soluções mais eficazes e sustentáveis para os desafios enfrentados pela tecnologia blockchain.

A tecnologia blockchain redefine de maneira fundamental a condução de transações, a assinatura de contratos e a realização de diversas operações, eliminando a necessidade de aprovações centralizadas para validação \cite{Blockchain}. Essa inovação estabelece um registro distribuído de transações, verificado por uma rede de computadores e organizado em blocos, criando uma cadeia de informações imutáveis e resistentes à adulteração \cite{IBM, Immutability}.

Embora as origens do blockchain remontem à década de 1990, quando Stuart e W. Scott Stornetta apresentaram a "árvore de carimbos de tempo" para garantir a integridade de registros digitais, foi apenas em 2008 que a tecnologia ganhou destaque com a publicação do artigo seminal de Satoshi Nakamoto sobre o Bitcoin \cite{Origem-Blockchain}.

Apesar das conquistas e inovações, o blockchain enfrenta desafios cruciais, incluindo escalabilidade, custos de transações, interoperabilidade, privacidade e segurança \cite{Scalability}. A escalabilidade, em particular, emerge como um desafio proeminente, limitando a quantidade de transações que as redes blockchain podem processar, o que impacta sua aplicabilidade em setores de alto volume, como sistemas de pagamento em larga escala \cite{Buterin2013}.

A avaliação da escalabilidade, geralmente medida em Transações por Segundo (TPS), destaca a importância do tamanho dos blocos e dos algoritmos de consenso, como o Proof-of-Work (PoW), na eficiência da rede. Enquanto blocos maiores podem aumentar a capacidade, há o risco de centralização e demanda por mais recursos de armazenamento. Além disso, considerações sobre privacidade e complexidade dos protocolos também influenciam a capacidade de processamento de transações \cite{Proof-of-Work}.

Este trabalho tem como objetivo explorar os limites do Blockchain em relação à escalabilidade e capacidade de processamento de transações em grandes escalas, utilizando métodos sistemáticos de análise de literatura. Apesar de ser uma tecnologia revolucionária com grande potencial para transformar diversos setores, o Blockchain apresenta desafios que podem afetar sua eficiência em larga escala.

A análise crítica concentra-se nos desafios inerentes à escalabilidade e capacidade de processamento em larga escala nas blockchains, examinando aspectos técnicos, como armazenamento, tempos de confirmação, processamento de nós e eficiência energética, enquanto explora questões de governança e adoção generalizada.

Para embasar essa análise crítica, apresentamos uma variedade de perspectivas sobre os desafios de escalabilidade no contexto das blockchains, destacando limitações atuais e explorando soluções propostas pela comunidade acadêmica e pela indústria. Essa abordagem busca informar e orientar futuros desenvolvimentos na busca por soluções mais eficazes e sustentáveis para os desafios enfrentados pela tecnologia blockchain.



%\bibliographystyle{alpha}
%\bibliography{sample}


%=====================================================

% o estilo de bibliografia é definido no arquivo packages.tex

% ATENÇÃO: evite usar \cite{}; prefira \citep{} e \citet{}

% base de bibliografia (BibTeX)

\bibliography{referencias}
\nocite{*}
%\bibliography{file1,file2,file3} % se tiver mais de um arquivo BibTeX

%=====================================================

\end{document}