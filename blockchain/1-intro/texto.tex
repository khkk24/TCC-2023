\chapter{Introdução}
\label{chapter_intro}


% A tecnologia blockchain tem revolucionado a forma como realizamos transações, assinaturas, contratos, etc., promovendo a descentralização e reduzindo a dependência de aprovações institucionais para validação \cite{Blockchain}. Ela oferece um registro distribuído de transações, compartilhado e verificado por uma rede de computadores  \cite{IBM}. Essas transações são organizadas em blocos, formando uma cadeia de informações imutáveis e protegidas contra adulteração \cite{Immutability}.

% Os primórdios do blockchain remontam à década de 1990, quando Stuart e W. Scott Stornetta introduziram o conceito de "árvore de carimbos de tempo" como uma solução para garantir a integridade de registros digitais. Contudo, foi somente em 2008 que a tecnologia ganhou notoriedade com a publicação do artigo "Bitcoin: A Peer-to-Peer Electronic Cash System" por Satoshi Nakamoto, descrevendo o funcionamento do Bitcoin, a primeira criptomoeda baseada em blockchain \cite{Origem-Blockchain}.

% Apesar de suas realizações e inovações, o blockchain enfrenta desafios significativos, como escalabilidade, custo de transações, interoperabilidade, privacidade e segurança  \cite{Scalability}. A escalabilidade, em particular, é um dos obstáculos mais prementes. A maioria das redes blockchain possui limitações na quantidade de transações processadas em um determinado período, o que pode dificultar sua adoção em áreas que demandam alto volume de transações, como sistemas de pagamento em larga escala \cite{Buterin2013}.

% A escalabilidade é frequentemente avaliada em termos de Transações por Segundo (TPS), uma métrica que avalia a capacidade de uma rede blockchain de processar transações em um segundo. O tamanho dos blocos e os algoritmos de consenso, como o Proof-of-Work (PoW), desempenham um papel crítico na escalabilidade. Blocos maiores podem aumentar a capacidade, mas também podem centralizar a rede e exigir mais recursos de armazenamento. Além disso, a privacidade das transações e os protocolos complexos também afetam a capacidade de processamento de transações \cite{Proof-of-Work}.

% Esta análise crítica se concentra nos desafios da escalabilidade e da capacidade de processamento em larga escala das blockchains. Examina as limitações técnicas, como armazenamento, tempos de confirmação, processamento de nós e eficiência energética . Além disso, discute as questões de governança e adoção em grande escala.

% Para embasar essa análise crítica, apresentamos diversos pontos de vista sobre os desafios da escalabilidade no contexto das blockchains. Destacamos as limitações atuais e exploramos soluções propostas pela comunidade acadêmica e pela indústria .

% A tecnologia blockchain tem redefinido fundamentalmente a maneira como conduzimos transações, assinamos contratos e realizamos uma variedade de operações, eliminando a necessidade de aprovações centralizadas para validação \cite{Blockchain}. Essa inovação estabelece um registro distribuído de transações, verificado por uma rede de computadores e organizado em blocos, criando uma cadeia de informações imutáveis e resistentes à adulteração \cite{IBM, Immutability}.

% Embora as origens do blockchain remontem à década de 1990, quando Stuart e W. Scott Stornetta apresentaram a "árvore de carimbos de tempo" para garantir a integridade de registros digitais, foi apenas em 2008 que a tecnologia ganhou destaque com a publicação do artigo seminal de Satoshi Nakamoto sobre o Bitcoin \cite{Origem-Blockchain}.

% Apesar das conquistas e inovações, o blockchain enfrenta desafios cruciais, incluindo escalabilidade, custos de transações, interoperabilidade, privacidade e segurança \cite{Scalability}. A escalabilidade, em particular, emerge como um desafio proeminente, limitando a quantidade de transações que as redes blockchain podem processar, o que impacta sua aplicabilidade em setores de alto volume, como sistemas de pagamento em larga escala \cite{Buterin2013}.

% A avaliação da escalabilidade, geralmente medida em Transações por Segundo (TPS), destaca a importância do tamanho dos blocos e dos algoritmos de consenso, como o Proof-of-Work (PoW), na eficiência da rede. Enquanto blocos maiores podem aumentar a capacidade, há o risco de centralização e demanda por mais recursos de armazenamento. Além disso, considerações sobre privacidade e complexidade dos protocolos também influenciam a capacidade de processamento de transações \cite{Proof-of-Work}.

% Este trabalho tem como objetivo explorar os limites do Blockchain em relação à escalabilidade e capacidade de processamento de transações em grandes escalas, utilizando métodos sistemáticos de análise de literatura. O Blockchain, apesar de ser uma tecnologia revolucionária com grande potencial para transformar diversos setores, apresenta alguns desafios que podem afetar sua eficiência em larga escala.
% Um dos principais desafios do Blockchain é a escalabilidade, que se refere à capacidade da tecnologia de lidar com um grande número de transações em um curto período de tempo. Isso ocorre porque cada transação é validada por um grande número de nós na rede, o que pode resultar em um gargalo de processamento de dados. Além disso, a capacidade de armazenamento de dados também pode ser um desafio, especialmente em Blockchains públicos, onde todas as transações são armazenadas em cada nó da rede.
% Esta análise crítica concentra-se nos desafios inerentes à escalabilidade e capacidade de processamento em larga escala nas blockchains. Examina aspectos técnicos, como armazenamento, tempos de confirmação, processamento de nós e eficiência energética, enquanto explora questões de governança e adoção generalizada.


% Para embasar essa análise crítica, apresentamos uma variedade de perspectivas sobre os desafios de escalabilidade no contexto das blockchains, destacando limitações atuais e explorando soluções propostas pela comunidade acadêmica e pela indústria. Essa abordagem busca informar e orientar futuros desenvolvimentos na busca por soluções mais eficazes e sustentáveis para os desafios enfrentados pela tecnologia blockchain.

A tecnologia blockchain redefine de maneira fundamental a condução de transações, a assinatura de contratos e a realização de diversas operações, eliminando a necessidade de aprovações centralizadas para validação \cite{Blockchain}. Essa inovação estabelece um registro distribuído de transações, verificado por uma rede de computadores e organizado em blocos, criando uma cadeia de informações imutáveis e resistentes à adulteração \cite{IBM, Immutability}.

Embora as origens do blockchain remontem à década de 1990, quando Stuart e W. Scott Stornetta apresentaram a "árvore de carimbos de tempo" para garantir a integridade de registros digitais, foi apenas em 2008 que a tecnologia ganhou destaque com a publicação do artigo seminal de Satoshi Nakamoto sobre o Bitcoin \cite{Origem-Blockchain}.

Apesar das conquistas e inovações, o blockchain enfrenta desafios cruciais, incluindo escalabilidade, custos de transações, interoperabilidade, privacidade e segurança \cite{Scalability}. A escalabilidade, em particular, emerge como um desafio proeminente, limitando a quantidade de transações que as redes blockchain podem processar, o que impacta sua aplicabilidade em setores de alto volume, como sistemas de pagamento em larga escala \cite{Buterin2013}.

A avaliação da escalabilidade, geralmente medida em Transações por Segundo (TPS), destaca a importância do tamanho dos blocos e dos algoritmos de consenso, como o Proof-of-Work (PoW), na eficiência da rede. Enquanto blocos maiores podem aumentar a capacidade, há o risco de centralização e demanda por mais recursos de armazenamento. Além disso, considerações sobre privacidade e complexidade dos protocolos também influenciam a capacidade de processamento de transações \cite{Proof-of-Work}.

Este trabalho tem como objetivo explorar os limites do Blockchain em relação à escalabilidade e capacidade de processamento de transações em grandes escalas, utilizando métodos sistemáticos de análise de literatura. Apesar de ser uma tecnologia revolucionária com grande potencial para transformar diversos setores, o Blockchain apresenta desafios que podem afetar sua eficiência em larga escala.

A análise crítica concentra-se nos desafios inerentes à escalabilidade e capacidade de processamento em larga escala nas blockchains, examinando aspectos técnicos, como armazenamento, tempos de confirmação, processamento de nós e eficiência energética, enquanto explora questões de governança e adoção generalizada.

Para embasar essa análise crítica, apresentamos uma variedade de perspectivas sobre os desafios de escalabilidade no contexto das blockchains, destacando limitações atuais e explorando soluções propostas pela comunidade acadêmica e pela indústria. Essa abordagem busca informar e orientar futuros desenvolvimentos na busca por soluções mais eficazes e sustentáveis para os desafios enfrentados pela tecnologia blockchain.
