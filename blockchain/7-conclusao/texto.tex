\chapter{Conclusão}

\label{chapter_conclusao}

A tecnologia blockchain emergiu como uma inovação transformadora, proporcionando descentralização e confiança em transações, contratos e registros digitais. Ao longo das décadas, desde sua concepção por Stuart e W. Scott Stornetta, até o aprimoramento por Satoshi Nakamoto com o advento do Bitcoin, a blockchain evoluiu significativamente. No entanto, apesar de suas conquistas, a tecnologia enfrenta desafios inerentes que exigem abordagens analíticas e soluções inovadoras.

A escalabilidade, custos de transação, interoperabilidade e privacidade destacam-se como desafios cruciais. A limitação na capacidade de processamento de transações em larga escala, muitas vezes avaliada em termos de Transações por Segundo (TPS), revela-se um obstáculo para a integração generalizada, especialmente em setores de alto volume, como sistemas de pagamento.

A modelagem analítica, utilizando ferramentas como cadeias de Markov, surge como uma abordagem valiosa para entender e superar esses desafios. Estudos, como aquele realizado por Cao et al. \cite{cao2019impact}, exploram a relação entre taxas de chegada de transações, peso cumulativo e atraso de confirmação. Esses modelos fornecem insights valiosos para otimizar o desempenho da blockchain.

Além disso, a análise crítica das limitações da tecnologia blockchain destaca a necessidade de abordagens multifacetadas. Soluções propostas pela comunidade acadêmica e da indústria incluem o uso de estruturas inovadoras, como o Tangle no IOTA, que elimina a necessidade de mineradores e oferece escalabilidade potencialmente ilimitada.

Diante desses desafios, a pesquisa contínua e a colaboração entre acadêmicos, desenvolvedores e profissionais da indústria são essenciais. A implementação bem-sucedida de modelos analíticos, como cadeias de Markov, combinada com soluções inovadoras, pode moldar o futuro da blockchain, tornando-a mais eficiente, segura e adaptável às demandas de uma variedade de setores.

Em resumo, a evolução da tecnologia blockchain, seus desafios e soluções propostas refletem um campo dinâmico e promissor. Ao enfrentar esses desafios de maneira colaborativa, podemos construir uma base sólida para o desenvolvimento contínuo da blockchain e sua integração eficaz em diversas aplicações.

