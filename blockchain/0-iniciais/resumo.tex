\begin{resumo}


    O Blockchain é uma tecnologia inovadora com o potencial de transformar diversos setores, desde serviços financeiros até cadeias de suprimentos. No entanto, sua adoção em larga escala tem enfrentado desafios significativos relacionados à escalabilidade e à capacidade de processamento de transações. A escalabilidade refere-se à capacidade do Blockchain de lidar com um grande número de transações e usuários sem comprometer a eficiência. À medida que mais participantes se juntam à rede, os tempos de confirmação de transações aumentam, tornando o Blockchain ineficiente em grande escala. O aumento do volume de transações também coloca pressão nos recursos computacionais necessários para validá-las. Outros desafios incluem o armazenamento contínuo de dados, à medida que o Blockchain cresce, e questões de governança e coordenação à medida que a rede envolve múltiplas organizações. Pesquisadores e desenvolvedores têm explorado soluções, como Blockchains de segunda camada, novos algoritmos de consenso (como Prova de Participação e Prova de Autoridade), e técnicas de dimensionamento horizontal. Este estudo aborda minuciosamente a problemática da escalabilidade e o desempenho do Blockchain, oferecendo uma análise aprofundada sobre como a escalabilidade representa um desafio premente nas redes blockchain. Além disso, são explorados os avanços tecnológicos que visam aprimorar e solucionar essa questão, destacando a importância de superar os obstáculos inerentes à expansão eficiente dessas redes.A escalabilidade do Blockchain é fundamental, e apesar dos desafios, o potencial disruptivo do Blockchain continua a atrair investimentos e inovações. À medida que a pesquisa e a colaboração avançam, soluções promissoras são esperadas para permitir que o Blockchain atinja seu pleno potencial em diversos setores.
    
    \end{resumo}
